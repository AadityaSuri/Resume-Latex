\documentclass[letterpaper,11pt]{article}

\usepackage{latexsym}
\usepackage[empty]{fullpage}
\usepackage{titlesec}
\usepackage{marvosym}
\usepackage[usenames,dvipsnames]{color}
\usepackage{verbatim}
\usepackage{enumitem}
\usepackage[hidelinks]{hyperref}
\usepackage{fancyhdr}
\usepackage[english]{babel}
\usepackage{tabularx}
\input{glyphtounicode}


%----------FONT OPTIONS----------
% sans-serif
% \usepackage[sfdefault]{FiraSans}
% \usepackage[sfdefault]{roboto}
% \usepackage[sfdefault]{noto-sans}
% \usepackage[default]{sourcesanspro}

% serif
% \usepackage{CormorantGaramond}
% \usepackage{charter}


\pagestyle{fancy}
\fancyhf{} % clear all header and footer fields
\fancyfoot{}
\renewcommand{\headrulewidth}{0pt}
\renewcommand{\footrulewidth}{0pt}

% Adjust margins
\addtolength{\oddsidemargin}{-0.5in}
\addtolength{\evensidemargin}{-0.5in}
\addtolength{\textwidth}{1in}
\addtolength{\topmargin}{-.5in}
\addtolength{\textheight}{1.0in}

\urlstyle{same}

\raggedbottom
\raggedright
\setlength{\tabcolsep}{0in}

% Sections formatting
\titleformat{\section}{
  \vspace{-4pt}\scshape\raggedright\large
}{}{0em}{}[\color{black}\titlerule \vspace{-5pt}]

% Ensure that generate pdf is machine readable/ATS parsable
\pdfgentounicode=1

%-------------------------
% Custom commands
\newcommand{\resumeItem}[1]{
  \item\small{
    {#1 \vspace{-2pt}}
  }
}

\newcommand{\resumeSubheading}[4]{
  \vspace{-2pt}\item
    \begin{tabular*}{0.97\textwidth}[t]{l@{\extracolsep{\fill}}r}
      \textbf{#1} & \textbf{#2} \\
      \textit{#3} & \textit{#4} \\
    \end{tabular*}\vspace{-7pt}
}

\newcommand{\resumeSubSubheading}[2]{
    \item
    \begin{tabular*}{0.97\textwidth}{l@{\extracolsep{\fill}}r}
      \textit{\small#1} & \textit{\small #2} \\
    \end{tabular*}\vspace{-7pt}
}

\newcommand{\resumeProjectHeading}[2]{
    \item
    \begin{tabular*}{0.97\textwidth}{l@{\extracolsep{\fill}}r}
      \small#1 & \textbf{#2} \\
    \end{tabular*}\vspace{-5pt}
}


\usepackage{calc} % added
\newcommand{\resumeSubSubheadingLeft}[2]{% % added
\begin{description}[leftmargin=!,labelwidth=\widthof{\small\bfseries #1}]
    \item[\small #1]{\textit{\small  #2}}
\end{description}
}

\newcommand{\resumeSubItem}[1]{\resumeItem{#1}\vspace{-4pt}}

\renewcommand\labelitemii{$\vcenter{\hbox{\tiny$\bullet$}}$}

\newcommand{\resumeSubHeadingListStart}{\begin{itemize}[leftmargin=0.15in, label={}]}
\newcommand{\resumeSubHeadingListEnd}{\end{itemize}}
\newcommand{\resumeItemListStart}{\begin{itemize}}
\newcommand{\resumeItemListEnd}{\end{itemize}\vspace{-5pt}}

%-------------------------------------------
%%%%%%  RESUME STARTS HERE  %%%%%%%%%%%%%%%%%%%%%%%%%%%%


\begin{document}

%----------HEADING----------
% \begin{tabular*}{\textwidth}{l@{\extracolsep{\fill}}r}
%   \textbf{\href{http://sourabhbajaj.com/}{\Large Sourabh Bajaj}} & Email : \href{mailto:sourabh@sourabhbajaj.com}{sourabh@sourabhbajaj.com}\\
%   \href{http://sourabhbajaj.com/}{http://www.sourabhbajaj.com} & Mobile : +1-123-456-7890 \\
% \end{tabular*}

\begin{center}
    \textbf{\Large Aaditya Suri} \\ \vspace{1pt}
    \small 236-990-5666 $|$ \href{mailto:aadityasuri01@gmail.com}{\underline{aadityasuri01@gmail.com}} $|$ 
    \href{https://www.linkedin.com/in/aadityasuri/}{\underline{Linkedin}} $|$
    \href{https://github.com/AadityaSuri}{\underline{Github}}
\end{center}


%-----------EDUCATION-----------
\section{Education}
  \resumeSubHeadingListStart
    \resumeSubheading
      {University of British Columbia}{\textnormal{Vancouver, BC}}
      {\textnormal{Bachelor of Applied Science – Computer Engineering}}{\textnormal{\textbf{Expected May 2024}}}

     \resumeSubSubheadingLeft% added <<<<<<<<<<<<<<<
    {Coursework:}{Software Construction I, Basic Algorithms and Data Structures, Computing Systems I, Computing Systems II, Digital Systems Design, Operating Systems, Software Construction II}
  \resumeSubHeadingListEnd

%
%-----------PROGRAMMING SKILLS-----------
\section{Technical Skills}
 \begin{itemize}[leftmargin=0.15in, label={}]
    \small{\item{
     \textbf{Languages}{: C/C++, Python, Java, SystemVerilog, C\#, MATLAB, Assembly, HTML, CSS, JavaScript} \\
     \textbf{Frameworks}{: MongoDB, Node.js, Flask, JUnit, Express, PyTorch, TensorFlow, Bootstrap, JSON} \\
     \textbf{Tools}{: Git, Google Cloud Platform, VS Code, PyCharm, IntelliJ, Unit Testing, ModelSim, Intel Quartus Prime, Oscilloscope, Raspberry Pi, Arduino} \\
     \textbf{Libraries}{: Pandas, NumPy, Matplotlib, OpenGL} \\
     \textbf{Concepts}{: RISC-V, MIPS, ARMv8, RTL Synthesis, TCP/IP, Multithreading, Linux, Operating Systems, FPGAs, MVC}
    }}
 \end{itemize}


%-------------------------------------------


%-----------EXPERIENCE-----------
\section{Experience}
  \resumeSubHeadingListStart
    % \resumeSubheading
    %   {Physical Design Engineer}{January 2023 -- Present}
    %   {Microchip Technology Inc.}{Burnaby, BC}
    %   \resumeItemListStart
    %     \resumeItem{}
    %   \resumeItemListEnd

    \resumeSubheading
      {Firmware Engineer}{February 2022 -- Present}
      {UBC Open Robotics}{Vancouver, BC}
      \resumeItemListStart
        \resumeItem{Used \textbf{C} to create and test the \textbf{firmware} and electrical design of a gripper for the RoboCup@Home project}
        \resumeItem{Calibrated various \textbf{sensors} to work with other electrical and mechanical components}
        \resumeItem{Programmed \textbf{microcontrollers} for making the gripper work in sync with the arms of the robot and the software}
      \resumeItemListEnd

    \resumeSubheading
      {Game Mechanics Programmer, Character Sprite Artist}{September 2020 -- April 2019}
      {UBC Game Development Club}{Vancouver, BC}
      \resumeItemListStart
        \resumeItem{Programmed game mechanics in \textbf{Unity} 2D game engine using \textbf{C\#} and \textbf{C++} }
        \resumeItem{Assess and troubleshoot computer problems brought by students, faculty and staff}
        \resumeItem{“Best Gameplay” and “Game of The Year” awarded to our game}
    \resumeItemListEnd

  \resumeSubHeadingListEnd


%-----------PROJECTS-----------
\section{Projects}
    \resumeSubHeadingListStart

        \resumeProjectHeading
          {\textbf{FPGA Rendering Accelerator} $|$ \emph{SystemVerilog, Intel Quartus}}{November 2022 -- December 2022}
          \resumeItemListStart
            \resumeItem{Implemented a digital circuit using \textbf{SystemVerilog} on the \textbf{DE1-SoC FPGA} to render the Mandelbrot set on a VGA screen using an optimized version of the escape time algorithm}
            \resumeItem{Designed multiple units to perform complex \textbf{fixed-point arithmetic}}
            \resumeItem{Improved circuit to be capable of continuously recalculating pixel values based on zoom level set by user}
          \resumeItemListEnd

          
      \resumeProjectHeading
          {\textbf{OS161} $|$ \emph{C, Assembly, Operating Systems}}{October 2022 -- December 2022}
          \resumeItemListStart
            \resumeItem{Designed the entire file and process management data structure and implemented multiple \textbf{syscalls} like open, close, read, write, fork, waitpid, exec, etc. using \textbf{C} and \textbf{Assembly}}
            \resumeItem{Added support to the OS to handle \textbf{concurrent threads} using \textbf{locks}, \textbf{semaphores}, and \textbf{condition variables}}
            \resumeItem{Implemented the entire \textbf{Virtual memory system} for the operating system}
          \resumeItemListEnd
    

        \resumeProjectHeading
            {\textbf{Online Chatting App} $|$ \emph{HTML, CSS, Javascript, MongoDB, Express}}{November 2022}
            \resumeItemListStart
                \resumeItem{Developed the entire Frontend and Backend for an online chatting app using \textbf{HTML}, \textbf{CSS}, and \textbf{JavaScript}}
                \resumeItem{Used the \textbf{Model-View-Controller(MVC)} software architecture for the application}
                \resumeItem{Added data persistance using \textbf{MongoDB}}
                \resumeItem{Added user authentication and middleware using \textbf{Express.js} to protect against \textbf{XSS attacks}}
            \resumeItemListEnd

        \resumeProjectHeading
            {\textbf{MaskMatcher} $|$ \emph{Python, PyTorch, Raspberry Pi}}{March 2022 -- April 2022}
            \resumeItemListStart
                \resumeItem{Developed a \textbf{machine learning} model for classification of faces wearing masks using the \textbf{PyTorch} library with over 80\% accuracy. Heavily optimized to run smoothly on \textbf{Raspberry Pi} using minimal hardware resources}
                \resumeItem{Used \textbf{OpenCV} and \textbf{Haar Cascades} to determine confidence interval of face mask detection}
                \resumeItem{Implemented multithreading on a \textbf{Raspberry Pi} to capture video from a webcam, run the \textbf{machine learning} model and live stream the video to a website simultaneously}
            \resumeItemListEnd
            
        \pagebreak

        \resumeProjectHeading
            {\textbf{Pianote} $|$ \emph{Python, PyGame}}{January 2022}
            \resumeItemListStart
                \resumeItem{Designed an app during a 24-hour \textbf{hackathon} which can be used by pianists for ear training }
                \resumeItem{Uses Python’s \textbf{Pygame} library for GUI and can play and recognize different notes played by the user on a virtual keyboard. Provides immediate feedback to user regarding the note they played and the expected note}
            \resumeItemListEnd

        \resumeProjectHeading
            {\textbf{Wikipedia Mediator Server} $|$ \emph{Java, Multithreading, Web Servers, JSON}}{December 2021}
            \resumeItemListStart
                \resumeItem{Setup a \textbf{multithreaded Java web server} capable of handling concurrent requests by parsing \textbf{JSON} input from multiple clients using the \textbf{Gson} library}
                \resumeItem{Integrated the \textbf{JWiki API} into program to serve data from Wikipedia as required by the client’s request}
                \resumeItem{Implemented a \textbf{BFS algorithm} to find minimum number of Wikipedia links required to get from one Wikipedia page to another which was able to produce more accurate results than already existing applications}
            \resumeItemListEnd

        \resumeProjectHeading
            {\textbf{RISC-V CPU} $|$ \emph{Verilog, RISC-V, ModelSim, Intel Quartus}}{October 2021 -- December 2021}
            \resumeItemListStart
                \resumeItem{Developed and implemented a Turing complete CPU based on \textbf{RISC-V computer architecture} using \textbf{Intel Quartus} on the \textbf{DE1-SoC FPGA} }
                \resumeItem{Simulated and debugged hardware modules in \textbf{ModelSim} by writing extensive Testbenches using \textbf{Verilog}}
                \resumeItem{Achieved up to 40\% reduction in instruction completion time by adding \textbf{pipelining hardware} to the design}
            \resumeItemListEnd

        \resumeProjectHeading
            {\textbf{Plagiarism Detection} $|$ \emph{Java, Google Cloud}}{October 2021}
            \resumeItemListStart
                \resumeItem{Analyzed text documents to report basic sentence and word level metrics using \textbf{Java}}
                \resumeItem{Integrated \textbf{Google Cloud} Natural Language client library into application to report positive or negative sentiment analysis of document }
                \resumeItem{Implemented the Jensen-Shannon Divergence method to analyze similarity between documents and detect plagiarism}
                \resumeItem{Used the \textbf{JUnit} framework to write comprehensive unit tests to debug and solve issues efficiently}
            \resumeItemListEnd
        
    \resumeSubHeadingListEnd



\end{document}
